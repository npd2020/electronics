\likechapter{Вступ}

У першій частині даної роботи нам було запропоновано познайомитися із роботою осцилографа Tektronix TDS 1002B, а саме: подати на вхід осцилографа довільний сигнал і отримати картинку на екрані, а також виконати Фур'є перетворення. Однак після цього етапу, завдання жорстоко ускладнювалися: потрібно було тепер вивести аж два сигнала на екран, та, увага, отримати картину на площині ХУ - тобто отримати фігури Ліссажу.

Другою частиною роботи, було ознайомлення з вимірювачем імпедансу HP 4192a. За допомогою даного приладу, ми досліджували залежність активного і реактивного опору резистора, ємності конденсатора та індуктивності котушки від частоти.