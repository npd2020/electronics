\chapter{Вимірювання імпедансу за допомогою HP4192a} 
\label{chap:first}

У цій частині ми вчились приборкувати \sout{дракона} імпедансметр HP4192a. У нас було 3 завдання: дослідити залежність імпедансу від частоти для резистора, конденсатора та котушки.

\section{Вимірювання активного та реактивного опору резистора}

Отримана експеримантальним шляхом залежність активного та реактивного опору резистора від частоти не викликає сумнівів. Справді, активний опір не залежав істотно від частоти, та був приблизно рівним $95.4 \Omega$ упродовж всього експерименту. На рахунок реактивного опору - при високих частотах опір явно росте, що викликано індуктивними властивостями дротів, використаних у цьому резисторі. При малих частотах реактивний опір також росте, однак це вже викликано тим, що наш резистор в деякому сенсі працює як конденсатор.

\begin{figure}[h]
    \begin{minipage}[h]{0.47\linewidth}
        \center{\includegraphics[width=1\linewidth]{pdf/c1_RR.pdf}} \\
    \end{minipage}
    \hfill
    \begin{minipage}[h]{0.47\linewidth}
        \center{\includegraphics[width=1\linewidth]{pdf/c1_RX.pdf}}\\
    \end{minipage}
    \caption{Залежність активного та реактивного опору резистора від частоти}
    \label{fig:part21}
\end{figure}


\section{Вимірювання активного та реактивного опору конденсатора}

Отримана експеримантальним шляхом залежність активного та реактивного опору конденсатора від частоти виявилась досить цікавою. Справді, активний опір досить швидко обвалюється, що природньо, однак при $\omega \approx 10^7 Hz$ має досить дивний пік, що можливо пов'язано із властивостями діалектрика всередині. Реактивний опір спочатку спадає, тобто поводить себе як в ідеальному конденсаторі, маючи мінімум при частоті $\omega \approx 10^5 Hz$, яка є власною резонансною частотою. Проте після цього опір починає збільшуватися, що пояснюється зростанням паразитної індуктивності, яка і визначає імпеданс конденсатора на високих частотах.

\begin{figure}[h]
    \begin{minipage}[h]{0.47\linewidth}
        \center{\includegraphics[width=1\linewidth]{pdf/c1_CR.pdf}} \\
    \end{minipage}
    \hfill
    \begin{minipage}[h]{0.47\linewidth}
        \center{\includegraphics[width=1\linewidth]{pdf/c1_CR2.pdf}}\\
    \end{minipage}
    \vfill
    \begin{minipage}[h]{0.47\linewidth}
        \center{\includegraphics[width=1\linewidth]{pdf/c1_CX.pdf}}\\
    \end{minipage}
    \caption{Залежність активного та реактивного опору конденсатора від частоти}
    \label{fig:part22}
\end{figure}

\section{Вимірювання активного та реактивного опору котушки}

Отримана експеримантальним шляхом залежність активного котушки від частоти викликає бажання у автора спитати викладача про доцільність даного експерименту. Справді, поведінка активного опору може бути охарактеризована прекрасною фразою брєд сивої кобили. Хоча пік реактивного опору в районі $\omega \approx 10^6 Hz$, після якого імпеданс котушки з індуктивного стає ємнісним виглядає досить цікавим. Пояснення цього явища виявилось досить простим: оскільки жодний фізичний прилад не можна вважати ідеальним, то котушка проявляє себе на певній частоті (її називають власною частотою даної котушки індуктивності) як схема RLC елементів.(рис. \ref{fig:ssanahuina}) Прийнято вважати, що ємність виникає між витками котушки і створює спостережуваний ефект, а опір пояснює певні втрати на даному елементі.\cite{huina}

\begin{figure}[h]
    \begin{minipage}[h]{0.47\linewidth}
        \center{\includegraphics[width=1\linewidth]{pdf/c1_LR.pdf}} \\
    \end{minipage}
    \hfill
    \begin{minipage}[h]{0.47\linewidth}
        \center{\includegraphics[width=1\linewidth]{pdf/c1_LX.pdf}}\\
    \end{minipage}    
    \caption{Залежність активного та реактивного опору котушки від частоти}
    \label{fig:part23}
\end{figure}
\begin{figure}[h]    
        \center{\includegraphics[width=0.5\linewidth]{contecv.png}} \\
    \caption{Представлення котушки як схеми RLC елементів}
    \label{fig:ssanahuina}
\end{figure}

Взагалі, робота з даним приладом варта 2 пострілам в голову, адже випадковим чином підібрані знаки, та інколи числа на його екрані не дають в спокійному режимі зняти потрібні покази.